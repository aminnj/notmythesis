\chapter{Summary and conclusions}

A sample of events with two same-sign or at least three charged leptons
produced in association with several jets in proton-proton collisions at
13\TeV, corresponding to an integrated luminosity of \sslumi, has been
studied to search for physics beyond the standard model, as well as for the
standard model (SM) production of four top quarks.

In the inclusive BSM analysis, no significant excesses were found and the
results are interpreted as limits on cross sections at 95\% confidence level
for the production of new particles in simplified supersymmetric models,
considering both R parity conserving and violating scenarios. The limits
are translated into lower mass limits that are as large as 2.1\TeV for
gluinos and 0.9\TeV for top and bottom squarks. 
To assist with future re-interpretations, model-independent limits are
provided as a function of the missing transverse momentum and the scalar sum
of jet transverse momenta in an event.

The SM four top quark production search analyzed the dataset using two
strategies, the first relying on a cut-based categorization in lepton
multiplicity, jet multiplicity, and jet flavor, and the second taking
advantage of a boosted decision tree (BDT) to distinguish the \tttt signal
from backgrounds. The more precise multivariate strategy yields an observed
(expected) significance of 2.6 (2.7) standard deviations relative to the
background-only hypothesis, and a measured value for the \tttt cross section
of $12.6^{+5.8}_{-5.2}\unit{fb}$. The results based on the two strategies are
in agreement with the SM prediction of $12.0^{+2.2}_{-2.5}\unit{fb}$. The
results of the BDT approach are used to constrain the top quark Yukawa
coupling, $y_{\PQt}$, resulting in the 95\% confidence level (\CL) limit of
$\abs{y_{\PQt}/y_{\PQt}^{\mathrm{SM}}} < 1.7$. The Higgs boson oblique
parameter in the effective field theory
framework~\cite{THEORY:ObliqueHiggs2019} is similarly constrained to $\hat{H}
< 0.12$ at 95\% \CL. Upper limits between 0.1 and 1.2 are also set on the
coupling between the top quark and a new scalar ($\phi$) or vector ($\cPZpr$)
particle with mass less than twice that of the top quark
($m_\PQt$)~\cite{THEORY:Alvarez2016nrz}. Considering new scalar or
pseudoscalar particles with $m > 2m_\PQt$, and decaying to \ttbar,
their production in association with one or two top quarks is probed.
The resulting cross section upper limit, between 15 and
35\unit{fb} at 95\% \CL, is interpreted in the contexts of Type-II
two-Higgs-doublet models and simplified dark matter models.

What a mouthful! In other words, the meta-summary is: the data collected by
the CMS detector from 2016 to 2018 did not turn up any interesting hints of
new physics in the same-sign final state. However, in the coming years, the
High-Luminosity LHC project seeks to at least double the dataset size
analyzed here. This will undoubtedly allow SM \tttt measurement to pass the
3-$\sigma$ evidence threshold, and maybe we will see hints of new physics.

