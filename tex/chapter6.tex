\chapter{Summary and conclusions}

\mytodo{need to merge}
\mytodo{add paragraph on future outlook}

\section{SUSY}

A sample of events with two same-sign or at least three charged leptons
(electrons or muons) produced in association with several jets in
proton-proton collisions at 13\TeV, corresponding to an integrated luminosity
of \sslumi, has been studied to search for manifestations of physics beyond
the standard model. The data are found to be consistent with the standard
model expectations. The results are interpreted as limits on cross sections
at 95\% confidence level for the production of new particles in simplified
supersymmetric models, considering both R parity conserving and violating
scenarios. Using calculations for these cross sections as functions of
particle masses, the limits are translated into lower mass limits that are as
large as 2.1\TeV for gluinos and 0.9\TeV for top and bottom squarks,
depending on the details of the model. Finally, to facilitate further interpretations of the search, model-independent
limits are provided as a function of the missing transverse momentum and the scalar sum of jet transverse momenta in an event.

\section{Four top}

The standard model (SM) production of \tttt has been studied in data from
$\sqrt{s} = 13\TeV$ proton-proton collisions collected using the CMS
detector during the LHC 2016--2018 data-taking period, corresponding to an
integrated luminosity of \sslumi. The final state with either two same-sign
leptons or at least three leptons is analyzed using two strategies, the
first relying on a cut-based categorization in lepton and jet multiplicity
and jet flavor, the second taking advantage of a multivariate approach to
distinguish the \tttt signal from its many backgrounds. The more precise
multivariate strategy yields an observed (expected) significance of 2.6
(2.7) standard deviations relative to the background-only hypothesis, and a
measured value for the \tttt cross section of $12.6^{+5.8}_{-5.2}\unit{fb}$.
The results based on the two strategies are in agreement with each other and
with the SM prediction of $12.0^{+2.2}_{-2.5}\unit{fb}$.

The results of the boosted decision tree (BDT) analysis are also used to
constrain the top quark Yukawa coupling $y_{\PQt}$ relative to its SM value.
based on the $\abs{y_{\PQt}}$ dependence of \xsectttt, resulting in the 95\% confidence
level (\CL) limit of $\abs{y_{\PQt}/y_{\PQt}^{\mathrm{SM}}} < 1.7$. The Higgs
boson oblique parameter in the effective field theory
framework~\cite{THEORY:ObliqueHiggs2019} is similarly constrained to $\hat{H} <
0.12$ at 95\% \CL. Upper limits ranging from 0.1 to 1.2 are also set on the
coupling between the top quark and a new scalar ($\phi$) or vector ($\cPZpr$)
particle with mass less than twice that of the top quark
($m_\PQt$)~\cite{THEORY:Alvarez2016nrz}. For new scalar ($\PH$) or
pseudoscalar ($\PSA$) particles with $m > 2m_\PQt$, and decaying to \ttbar,
their production in association with a single top quark or a top quark pair
is probed. The resulting cross section upper limit, between 15 and
35\unit{fb} at 95\% \CL, is interpreted in the context of Type-II
two-Higgs-doublet
models as a
function of $\tan \beta$ and $m_\mathrm{\PH/\PSA}$, and in the context of
simplified dark matter models, as a
function of $m_\mathrm{\PH/\PSA}$ and the mass of the dark matter candidate.
