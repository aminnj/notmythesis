\chapter{Introduction}


\section{Standard model}

\mytodo{describe SM}
\mytodo{table of particles}

\section{Beyond the standard model}

\mytodo{issues with sm, higgs mass, hierarchy problem}
\mytodo{dark mass, neutrino masses, matter asymmetry}
\mytodo{SUSY}

Stable LSP means conservation of symmetry called R-parity
$P_R=(-1)^{3B+L+2s}$ where $B$ is baryon number, $L$ is lepton number,
and $s$ is spin.

\section{LHC}
\mytodo{todo}

% https://twiki.cern.ch/twiki/bin/viewauth/CMS/Internal/PubDetector

LHC~\cite{LHC:Evans1129806}
CMS~\cite{CMS:Chatrchyan2008zzk,CMS:PTDR2}
Tracker~\cite{CMS:TRK11001,CMS:Dominguez1481838}
ECAL~\cite{CMS:Khachatryan2015hwa}
HCAL~\cite{CMS:PTDR2}
Muon~\cite{CMS:Sirunyan2018fpa}
Trigger~\cite{CMS:Khachatryan2016bia}
PF~\cite{CMS:PRF14001}

$N = \sigma \cdot L_\text{int}$

instantaneous luminosity is on the order of $10^{34} \unit{cm}^{-2} \unit{s}^{-1}$.
The metric unit for a area (b/barn), commonly used to express cross sections and luminosities,
is equivalent to $10^{-24} \unit{cm}^2$. Thus, the LHC instantaneous luminosity can
be roughly written as $1 \unit{fb}^{-1} \unit{day}^{-1}$


\section{CMS}
\mytodo{todo}

The Compact Muon Solenoid (CMS) detector is one of the general purpose detectors
situated on the LHC ring.

\subsection{Tracker}
\mytodo{todo}

% pg 15 of http://cds.cern.ch/record/1481838/files/CMS-TDR-011.pdf
3 layers ranging between 4.4cm and 10.2cm (2016)
4 layers ranging between 3.0cm and 16.0cm (2017-2018)

% Just outside the beampipe within which collisions occur

\subsection{Electromagnetic calorimeter}
\mytodo{todo}

\subsection{Hadronic caolorimeter}
\mytodo{todo}

\subsection{Muon system}
\mytodo{todo}

\subsection{Trigger and reconstruction}
\mytodo{todo, PF}


The remaining sections of the thesis are structured as follows.
Chapter~\ref{chap:ssdl} introduces the same-sign dilepton final state,
a promising avenue to explore new and rare physics, and its
SM and BSM physical sources. Chapter~\ref{chap:objects} discusses analysis
objects and selections for two related analyses: one is an inclusive search
for BSM, focusing on a variety of SUSY signals; the second is a SM search for the production
for four top quarks. Chapter~\ref{chap:backgrounds} describes the
SM backgrounds to these analyses. Chapter~\ref{chap:results} presents the
final results for each, and a summary with concluding remarks is given in
Chapter~\ref{chap:summary}. The results and techniques presented throughout
this thesis correspond to the published results in
Refs.~\cite{CMS:myTOPRun2,CMS:mySUSRun2PAS,CMS:myTOP2016,CMS:mySUS2016}.