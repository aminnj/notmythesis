\chapter{Introduction}


\section{Standard model}
\mytodo{todo}

% \mytodo{describe SM}
% \mytodo{table of particles}

\section{Beyond the standard model}
\mytodo{todo}

% \mytodo{issues with sm, higgs mass, hierarchy problem}
% \mytodo{dark mass, neutrino masses, matter asymmetry}
% \mytodo{SUSY}
% Stable LSP means conservation of symmetry called R-parity
% $P_R=(-1)^{3B+L+2s}$ where $B$ is baryon number, $L$ is lepton number,
% and $s$ is spin.

\section{LHC}
\mytodo{todo}

% https://twiki.cern.ch/twiki/bin/viewauth/CMS/Internal/PubDetector

LHC~\cite{LHC:Evans1129806}

% $N = \sigma \cdot L_\text{int}$
% instantaneous luminosity is on the order of $10^{34} \unit{cm}^{-2} \unit{s}^{-1}$.
% The metric unit for a area (b/barn), commonly used to express cross sections and luminosities,
% is equivalent to $10^{-24} \unit{cm}^2$. Thus, the LHC instantaneous luminosity can
% be roughly written as $1 \unit{fb}^{-1} \unit{day}^{-1}$


\section{CMS}
\mytodo{todo}

CMS~\cite{CMS:Chatrchyan2008zzk,CMS:PTDR2}
% Tracker~\cite{CMS:TRK11001,CMS:Dominguez1481838}
% ECAL~\cite{CMS:Khachatryan2015hwa}
% HCAL~\cite{CMS:PTDR2}
% Muon~\cite{CMS:Sirunyan2018fpa}
% Trigger~\cite{CMS:Khachatryan2016bia}
% PF~\cite{CMS:PRF14001}


% The Compact Muon Solenoid (CMS) detector is one of the general purpose detectors
% situated on the LHC ring.

transverse momentum (\pt).

pseudorapidity $\eta \equiv -\ln (\tan (\theta/2))$

Let us now turn to the individual subsystems that comprise the CMS detector
starting from the center, working our way radially outward. A cutaway view of
the CMS detector, highlighting these subsystems, is shown in
Figure~\ref{fig:cmsdetector}.

\begin{figure}[!htbp]
    \centering
    \includegraphics[width=0.99\linewidth]{figs/misc/cms.png}
    \caption{
        A cutaway view of the CMS detector.
    }
    \label{fig:cmsdetector}
\end{figure}

\FloatBarrier

\subsection{Tracker}

The innermost subdetector is the
tracker~\cite{CMS:TRK11001,CMS:Dominguez1481838}, which is composed of an
inner pixel detector and an outer strip tracker. Both use silicon technology
which allows for the formation of electron-hole pairs when a charged particle
passes through.
There are 3 layers of inner
pixel detectors with a transverse radial position ranging between $4.4\unit{cm}$ and $10.2\unit{cm}$. In 2017,
these were replaced with 4 layers ranging between $3.0\unit{cm}$ and $16.0\unit{cm}$.
The outer strip tracker consists of 10 layers in the central barrel region, extending
out to $1.1\unit{m}$. The endcaps consist of 2 disks for the inner pixel detector
and 12 disks in the strip tracker.

The tracker allows for precise reconstruction of particle trajectories and 
vertexing up to $|\eta|=2.5$ and for full range of azimuthal angles.

% % pg 15 of http://cds.cern.ch/record/1481838/files/CMS-TDR-011.pdf
% 3 layers ranging between 4.4cm and 10.2cm (2016)
% 4 layers ranging between 3.0cm and 16.0cm (2017-2018)

\subsection{Electromagnetic calorimeter}

Sitting just outside of the tracker is the electromagnetic calorimeter (ECAL).
The ECAL is composed of nearly 80,000 lead tungstate scintillating crystals tiled to form a cylinder,
and is split into a barrel section and two endcap sections~\cite{CMS:Khachatryan2015hwa}.
Together, these sections provide full azimuthal coverage and coverage up to $|\eta|=3.0$.

The high-density lead tungstate crystals, which are $22\unit{cm}$ in depth
and point toward the center of the CMS detector,
have a radiation length of $0.85\unit{cm}$. Consequently, the 25 radiation
lengths in each crystal allow electromagnetic showers to be almost completely
longitudinally contained within the single layer of crystals. The small
Moli\'ere radius of the crystals ($2.2\unit{cm}$), which coincides with the
width of the crystals themselves ($2.2\unit{cm}$ in the barrel and
$2.9\unit{cm}$ in the endcaps), ensures that the transverse profile of
showers is mostly contained within just a single crystal.

\subsection{Hadronic calorimeter}

Outside of the ECAL is the hadronic calorimeter (HCAL)~\cite{CMS:PTDR2},
which is split into four sets of calorimeters to cover different geometrical
regions: barrel (HB), endcap (HE), outer (HO), and forward (HF) calorimeters.
Similar to the ECAL, the first three sections collectively coverage of hadronic showers 
up to $|\eta|=3.0$, and the HF extends the coverage to $|\eta|=5.0$.

The sampling calorimeters of the HCAL are made of alternating layers of
absorber (brass or steel) that induce hadronic showers, and plastic
scintillators. In the barrel, the brass absorbers have a total of nearly 6
interaction lengths, and along with the plastic scintillators, have
transverse segmentation with $\eta$ and $\phi$ widths of 0.087.

\subsection{Muon system}

Last, but not least, is the namesake muon system~\cite{CMS:Sirunyan2018fpa} outside of the calorimeters
and interspersed between parts of the steel return yoke. The muon system
uses three main technologies: Drift Tubes (DTs) in the barrel, Cathode
Strip Chambers (CSCs) in the endcaps, and Resistive Plate Chambers (RPCs)
in the barrel and endcaps. Together, the muon system covers up to $|\eta|=2.4$.

The barrel DTs are arranged into four stations of concentric cylinders,
and use an ionizing gas mixture of Ar/CO${}_2$ and sensitive gold-plated steel wires
to detect the ionization of throughgoing muons. Each DT has two or three layers
with wires that are either perpendicular to the beam line (to measure the $z$ coordinate),
or parallel (to measure the $\phi$ coordinate). Cylindrical stacking of
DTs provides a measurement of the $r$ coordinate.

The trapezoidal endcap CSCs cover trajectories at high-$|\eta|$. Each
detector is a multiwire proportional chamber with 6 anode wire planes alternated
with 7 cathode strip panels. Wires are oriented azimuthally and allow the measurement
of a trajectory's $r$ coordinate. Strips are oriented radially and precisely measure
the $\phi$ coordinate. CSCs provide a spatial resolution on the order of $100~\mathrm{\mu m}$.

While not as spatially precise as DTs and CSCs, the gas-filled parallel-plate RPC detectors provide excellent
time resolution (several nanoseconds) to help tag the time/bunch crossing of muon hits.

\subsection{Trigger and reconstruction}

With the extremely high rate of collisions delievered to the CMS detector by the LHC,
it is not feasible to store and reconstruct collision information for every event.
A trigger system provides a solution to this problem by evaluating if
an event is potentially useful enough to keep. This is implemented
as a two-tiered system: a Level-1 (L1) trigger and a subsequent High-Level Trigger (HLT)~\cite{CMS:Khachatryan2016bia}.

The L1 trigger stage uses custom hardware with FPGAs and look up tables to
roughly calculate various particle momenta and positions based
on information from different subdetectors
as fast as possible, in order to evaluate if an event passes specific quality
criteria. The L1 trigger reduces the event rate by a factor of a million, and is designed
to output events at less than 100 kHz.

After an event passes the L1 trigger, it is passed through the HLT stage
which is implemented in pure software and runs on a farm of computers.
Events are partially reconstructed with more precision than with the L1
stage. This HLT stage reduces the event rate to less than 1 kHz. At this point,
events are passed along to be fully reconstructed and persisted at storage facilities.

CMS uses particle-flow (PF) reconstruction algorithms~\cite{CMS:PRF14001} to
utilize information from all subdetectors to reconstruct particles within an
event. At the lowest level, particle trajectories are reconstructed from hits
in sensitive layers of detectors. Reconstruction of certain particles can
then be loosely identified with different subdetectors. For example,
electrons and photons are reconstructed with information from the tracker and
ECAL. Charged and neutral hadrons use information from the ECAL and HCAL.
Muons are reconstructed with information from the tracker and muon system.
Neutrinos are not reconstructed as they pass through the detector without
interacting, but their presence can be inferred from an imbalance of energy
in an event.
$\newline$

The remaining sections of the thesis are structured as follows.
Chapter~\ref{chap:ssdl} introduces the same-sign dilepton final state,
a promising avenue to explore new and rare physics. 
Chapter~\ref{chap:objects} discusses analysis
objects and selections for two related analyses: one is an inclusive search
for BSM, focusing on a variety of SUSY signals; the second is a SM search for the production
for four top quarks. Both utilize data delivered by the LHC
and collected by the CMS detector from 2016--2018, totaling $137$~\fbinv. 
Chapter~\ref{chap:backgrounds} describes the
SM backgrounds to these analyses. Chapter~\ref{chap:results} presents the
final results for each, and a summary with concluding remarks is given in
Chapter~\ref{chap:summary}. The results and techniques presented throughout
this thesis correspond to the published results in
Refs.~\cite{CMS:myTOPRun2,CMS:mySUSRun2PAS,CMS:myTOP2016,CMS:mySUS2016}.