\chapter{Same-sign dilepton final state}
\mytodo{make todo}

\begin{figure}[!hbtp]
\centering
\includegraphics[width=.95\textwidth]{figs/misc/sm_xsecs.pdf} \\
\caption{Summary of SM cross section measurements at CMS~\cite{CMS:SMxsecs}.}
\label{fig:SMxsecs}
\end{figure}


\section{SUSY processes}
\mytodo{reword, restructure}
\mytodo{put plot of SUSY xsecs}

\begin{table} [h!]
\begin{center}
\resizebox{0.99\textwidth}{!}{
{\renewcommand{\arraystretch}{1.7}
\begin{tabular}{l|p{0.4\textwidth}cccc}
\hline
Model & Process & Constraint & Mass 1 & Mass 2 & RPV? \\
\hline
\Totttt & $\gluino\to\ttbar\lsp$ & \NA & $m_{\gluino}$ & $m_{\lsp}$ & \\
\TfttbbWW & $\gluino\to\cPqt\cPqb\chiplmin$ & $m_{\chiplmin}=m_{\lsp}+5\GeV$ & $m_{\gluino}$ & $m_{\lsp}$ & \\
\Tftttt & $\gluino\to\susytopone\cPqt$, $\susytopone\to\cPqt\lsp$ & $m_{\susytop}-m_{\lsp}=m_{\PQt}$ & $m_{\gluino}$ & $m_{\lsp}$ & \\
\Tfttcc & $\gluino\to\susytopone\cPqt$, $\susytopone\to\cPqc\lsp$ & $m_{\susytop}-m_{\lsp}=20\GeV$&  $m_{\gluino}$ & $m_{\lsp}$ & \\
\TfqqqqWW & $\gluino\to\cPq\cPq'\chiplmin$, $\chiplmin\to\PW^{\pm}\lsp$ & $m_{\chiplmin}=0.5(m_{\gluino}+m_{\lsp})$ & $m_{\gluino}$ & $m_{\lsp}$ & \\
\TfqqqqWW & $\gluino\to\cPq\cPq'\chiplmin$, $\chiplmin\to\PW^{\pm}\lsp$ & $m_{\chiplmin}=m_{\lsp}+20\GeV$ & $m_{\gluino}$ & $m_{\lsp}$ & \\
\TfqqqqWZ & $\gluino\to\cPq\cPq'(\chiplmin/\neutralinotwo)$, \newline $\chiplmin\to\PW^{\pm}\lsp$, $\neutralinotwo\to\PZ\lsp$ & $m_{\chiplmin}=0.5(m_{\gluino}+m_{\lsp})$ & $m_{\gluino}$ & $m_{\lsp}$ & \\
\TfqqqqWZ & $\gluino\to\cPq\cPq'(\chiplmin/\neutralinotwo)$, \newline $\chiplmin\to\PW^{\pm}\lsp$, $\neutralinotwo\to\PZ\lsp$ & $m_{\chiplmin}=m_{\lsp}+20\GeV$ & $m_{\gluino}$ & $m_{\lsp}$ & \\
\TsttWW & $\sbottomone\to\cPqt\chiplmin$ & $m_{\lsp}=50\GeV$ & $m_{\sbottomone}$ & $m_{\chiplmin}$ & \\
\TsttHZ & $\susytoptwo\to\susytopone\PH$, $\susytopone\to\cPqt\lsp$ & $m_{\susytopone}-m_{\lsp}=175\GeV$ & $m_{\susytoptwo}$ & $m_{\susytopone}$ & \\
\TsttHZ & $\susytoptwo\to\susytopone(\PH/\PZ)$, $\susytopone\to\cPqt\lsp$ & $m_{\susytopone}-m_{\lsp}=175\GeV$&  $m_{\susytoptwo}$ & $m_{\susytopone}$ & \\
\TsttHZ & $\susytoptwo\to\susytopone\PZ$, $\susytopone\to\cPqt\lsp$ & $m_{\susytopone}-m_{\lsp}=175\GeV$ & $m_{\susytoptwo}$ & $m_{\susytopone}$ & \\
\ToqqqqL & $\gluino\to\cPq\cPq\bar{\cPq}\bar{\cPq}+e/\mu/\tau$ & \NA & $m_{\gluino}$ & \NA & Yes \\
\Totbs & $\gluino\to\cPqt\cPqb\cPqs$ & \NA & $m_{\gluino}$ & \NA & Yes \\
\hline
\end{tabular}}}
\caption{Summary of simplified SUSY models considered in this analysis. The third
and second from last columns give the one/two masses which are scanned over. The last column
marks processes with R parity violation.}
\label{tab:susyprocesses}
\end{center}
\end{table}



Gluino pair production models giving rise to signatures with up to four \cPqb\
quarks and up to four \PW\ bosons are shown in Fig.~\ref{fig:susy_diag_set1}. In
these models, the gluino decays to the lightest squark ($\gluino \to \susyq
\cPq$), which in turn decays to same-flavor ($\susyq \to \cPq \lsp$) or
different-flavor ($\susyq \to \cPq' \chiplmin$) quarks. The chargino
($\chiplmin$) decays to a \PW\ boson and a neutralino ($\lsp$) via $\chiplmin
\to \PW^{\pm} \lsp$, where the \lsp is taken to be the lightest stable SUSY
particle and escapes detection.

The first scenario, denoted by \Totttt and displayed in
Fig.~\ref{fig:susy_diag_set1}a, includes an off-shell top squark ($\susytop$)
leading to the three-body decay of the gluino, $\gluino \to \ttbar\lsp$,
resulting in events with four \PW\ bosons and four \cPqb\ quarks.
Figure~\ref{fig:susy_diag_set1}b presents a similar model (\TfttbbWW) where the
gluino decay results in a chargino that further decays into a neutralino and
a \PW\ boson. The model shown in Fig.~\ref{fig:susy_diag_set1}c (\Tftttt) is the
same as \Totttt except that the intermediate top squark is on-shell. The mass
splitting between the $\susytop$ and the \lsp is taken to be $m_{\susytop} -
m_{\lsp} = m_{\cPqt}$, where $m_{\cPqt}$ is the top quark mass. This choice
maximizes the kinematic differences between this model and \Totttt, and also
corresponds to one of the most challenging regions of parameter space for the
observation of the $\susytop \to \cPqt \lsp$ decay since the neutralino is
produced at rest in the top squark rest frame. The decay chain of
Fig.~\ref{fig:susy_diag_set1}d (\Tfttcc) is identical to that of \Tftttt except
that the $\susytop$ decay involves a \PQc quark. In
Fig.~\ref{fig:susy_diag_set1}e, the decay process includes a virtual light-flavor
squark, leading to three-body decays of $\gluino \to \cPq \cPq' \chiplmin$ or
$\gluino \to \cPq \cPq' \neutralinotwo$, with a resulting signature of two
\PW\ bosons, two \PZ\ bosons, or one of each (the case shown in
Fig.~\ref{fig:susy_diag_set2}e), and four light-flavor jets. This model,
\TfqqqqWZ, with a resulting signature of one \PW\ boson and one \PZ\ boson,
is studied with two different assumptions for the chargino mass:
$m_{\chiplmin} = 0.5(m_{\gluino} + m_{\lsp})$, and $m_{\chiplmin} =
m_{\lsp}+20 \GeV$, producing on- and off-shell bosons, respectively. The
model is also considered with the assumption of decays to two \PW\ bosons
exclusively (\TfqqqqWW).

Figure~\ref{fig:susy_diag_set2}a shows a model of bottom squark production with
subsequent decay of $\sbottomone \to \cPqt \chiplmin$, yielding two \cPqb\
quarks and four \PW\ bosons. This model, \TsttWW, is considered as a function
of the the lightest bottom squark, $\sbottomone$, and \chiplmin masses. The
\lsp mass is fixed to be 50\GeV, causing two of the \PW\ bosons to be
produced off-shell when the \chiplmin mass is less than approximately
130\GeV. Figure~\ref{fig:susy_diag_set2}b displays a model similar to \TsttWW, but
with top squark pair production and a subsequent decay of
$\susytoptwo\to\susytopone\PH/\PZ$, with $\susytopone \to \cPqt\lsp$,
producing signatures with two \PH\ bosons, two \PZ\ bosons, or one of each.
In this model, \TsttHZ, the \lsp mass is fixed such that
$m(\susytopone)-m(\lsp)=m_{\cPqt}$.

The R parity violating decays considered in this analysis are \ToqqqqL
(Fig.~\ref{fig:susy_diag_set3}a) and \Totbs (Fig.~\ref{fig:susy_diag_set3}b). In
\ToqqqqL, the gluino decays to the lightest squark ($\gluino \to \susyq
\cPq$), which in turn decays to a quark ($\susyq \to \cPq \lsp$), but decays
with the $\lsp$ off shell (violating R parity) into two quarks and a charged
lepton, giving rise to a prompt 5-body decay of the gluino. In \Totbs, each
gluino decays into three different SM quarks (a top, a bottom, and a strange
quark).

\begin{figure}[htb!]
    \centering
    \subfloat[T1tttt]{\includegraphics[width=0.48\textwidth]{figs/ssp/diag_T1tttt}}
    \subfloat[T5ttbbWW]{\includegraphics[width=0.48\textwidth]{figs/ssp/diag_T5ttbbWW}} \\
    \subfloat[T5tttt]{\includegraphics[width=0.48\textwidth]{figs/ssp/diag_T5tttt}}
    \subfloat[T5ttcc]{\includegraphics[width=0.48\textwidth]{figs/ssp/diag_T5ttcc}} \\
    \subfloat[T5qqqqWZ]{\includegraphics[width=0.48\textwidth]{figs/ssp/diag_T5qqqqWZ}}
\caption{Diagrams illustrating the simplified RPC SUSY models with gluino production considered in this analysis.}
\label{fig:susy_diag_set1}
\end{figure}

\begin{figure}[htb!]
    \centering
    \subfloat[T6ttWW]{\includegraphics[width=0.48\textwidth]{figs/ssp/diag_T6ttWW}}
    \subfloat[T6ttHZ]{\includegraphics[width=0.48\textwidth]{figs/ssp/diag_T6ttHZ}} \\
\caption{Diagrams illustrating the simplified RPC SUSY models with squark production considered in this analysis.}
\label{fig:susy_diag_set2}
\end{figure}

\begin{figure}[htb!]
    \centering
    \subfloat[T1qqqqL]{\includegraphics[width=0.53\textwidth]{figs/ssp/diag_T1qqqqL}}
    \subfloat[T1tbs]{\includegraphics[width=0.45\textwidth]{figs/ssp/diag_T1tbs}} \\
\caption{Diagrams illustrating the two simplified RPV SUSY models considered in this analysis.}
\label{fig:susy_diag_set3}
\end{figure}

\begin{figure}[htb!]
    \centering
    \includegraphics[width=0.75\textwidth]{figs/ssan/plot_susy_xsecs}
\caption{Strong production cross sections for SUSY processes at the LHC. Calculations from \cite{THEORY:SUSYxsecs}}.
\label{fig:susy_xsecs}
\end{figure}


\section{Contributions to \tttt production}

\subsection{On-shell}
\mytodo{2hdm, dark matter}

\begin{figure}[htb!]
    \centering
    \subfloat[]{\includegraphics[width=0.33\textwidth]{figs/ftan/bsm_tth_diagram}}
    \subfloat[]{\includegraphics[width=0.33\textwidth]{figs/ftan/bsm_thw_diagram}}
    \subfloat[]{\includegraphics[width=0.33\textwidth]{figs/ftan/bsm_thq_diagram}} \\
\caption{BLAH}
\label{fig:thdm_diagrams}
\end{figure}


\begin{figure}[htb!]
    \centering
    \includegraphics[width=0.75\textwidth]{figs/ftan/bsm_dm_diagrams}
\caption{BLAH}
\label{fig:dm_diagrams}
\end{figure}

\begin{figure}[htb!]
    \centering
    \includegraphics[width=0.75\textwidth]{figs/ftan/plot_1d_2hdm_xsec}
\caption{BLAH}
\label{fig:thdm_1d_xsec}
\end{figure}

\begin{figure}[htb!]
    \centering
    \subfloat[scalar]{\includegraphics[width=0.50\textwidth]{figs/ftan/plot_2d_2hdm_xsec_h}}
    \subfloat[pseudoscalar]{\includegraphics[width=0.50\textwidth]{figs/ftan/plot_2d_2hdm_xsec_a}}
\caption{BLAH}
\label{fig:thdm_2d_xsecs}
\end{figure}

\begin{figure}[htb!]
    \centering
    \subfloat[scalar]{\includegraphics[width=0.50\textwidth]{figs/ftan/plot_2d_dmscalar_xsec_totsm}}
    \subfloat[pseudoscalar]{\includegraphics[width=0.50\textwidth]{figs/ftan/plot_2d_dmpseudo_xsec_totsm}}
\caption{BLAH}
\label{fig:dm_2d_xsecs}
\end{figure}


\subsection{Off-shell}

\subsubsection{Top quark yukawa coupling}

The SM $pp \rightarrow \tttt$ process includes diagrams with virtual Higgs bosons,
as shown in Fig.~\ref{fig:feynYukawa}. 
The amplitude corresponding to these diagrams is 
proportional to the square of the top Yukawa coupling,
and thus, the cross section of SM \tttt provides
a promising probe into the top Yukawa coupling.

\begin{figure}[!hbtp]
\centering
\includegraphics[width=.35\textwidth]{figs/ftp/ftdiag3.pdf} \\
\caption{One of the Feynman diagrams for \tttt including a virtual Higgs.}
\label{fig:feynYukawa}
\end{figure}

Using the notation of Reference~\cite{THEORY:TopYukawaTTTT} the \tttt cross section can be written 
as 

\begin{equation} 
\label{eq:yukawa}
\sigma(\tttt) = \sigma^{\text{SM}}(\tttt)_{g+Z/\gamma} + k_t^4 \sigma^{\text{SM}}(\tttt)_H + k_t^2 \sigma^{\text{SM}}_{\rm int}
\end{equation} 

\noindent where $k_t \equiv y_t/y_t^{\text{SM}}$, $y_t$ is the top Yukawa coupling, and $y_t^{\text{SM}}$ is its SM value.
In equation~\ref{eq:yukawa} the first term on the right hand side corresponds to the 
SM contribution to the cross section from diagrams with gluons or $Z/\gamma$, the second term
is the contribution from diagrams with virtual Higgs bosons, and the third term is the interference between
the two previous terms. Therefore, given a theoretical calculation and a measurement of $\sigma(\tttt)$, one can put 
constraints on $|y_t/y_t^{\text{SM}}|$.

The authors of Reference~\cite{THEORY:TopYukawaTTTT} have calculated the cross section terms at LO.
These are given in Table~\ref{tab:yukawa} and are shown in Fig.~\ref{fig:cross_section_yt},
where the figure shows a curve normalized such that the prediction matches the NLO calculation of 
the \tttt cross section of $12.0^{+2.2}_{-2.5}\unit{fb}$ calculated in Ref.~\cite{THEORY:Frederix2017wme}.
The upper and lower values given in Table~\ref{tab:yukawa} correspond to variations
of the renormalization and factorization scale up and down by a factor of two, respectively.

\begin{table} [h!]
\begin{center}
{\renewcommand{\arraystretch}{1.3}
\begin{tabular}{l|ccc}
\hline
   & lower & central & upper \\
\hline
$ \sigma^{\text{SM}}(\tttt)_{g+Z/\gamma}  $ & 14.104 fb & 9.997 fb &  6.378 fb \\
$ \sigma^{\text{SM}}(\tttt)_H $                  & 1.625 fb &  1.167 fb &  0.7655 fb \\
$\sigma^{\text{SM}}_{\rm int} $                   & -2.152 fb &  -1.547 fb &  -0.999 fb \\
\hline
\end{tabular}}
\caption{LO calculation of the terms in equation~\ref{eq:yukawa} from
Reference~\cite{THEORY:TopYukawaTTTT}.  
The uncertainties are from private communications with the authors.}
\label{tab:yukawa}
\end{center}
\end{table}


\begin{figure}[!htbp]
    \centering
    \includegraphics[width=0.75\linewidth]{figs/ftan/cross_section_yt.pdf}
    \caption{
        Predicted \tttt cross section as a function of $|y_t/y_t^{\text{SM}}|$.
    }
    \label{fig:cross_section_yt}
\end{figure}

\subsubsection{Light off-shell mediators}

The production of \tttt may also be influenced by a neutral scalar mediator
($\phi$) or neutral vector mediator ($Z'$) which couple to top quarks and have
masses less than twice the mass of the top quark, distinguishing them from
from similar processes within the 2HDM framework, for example. The off-shell contributions
to the SM \tttt production can be large, as shown in
Ref.~\cite{THEORY:Alvarez2016nrz}. For a large range of masses, the authors have
shown that kinematics are identical when considering these additional
processes, so that the total \tttt cross section is subject to a simple
rescaling.  Corresponding coupling terms in the lagrangian of the form
\begin{equation}
    \mathcal{L}_{Z'}=-g_{t Z'}\bar{t}_R \slashed{Z}' t_R
    \quad\quad\quad
    \mathcal{L}_{\phi}=-g_{t \phi}\bar{t}_L \phi t_R
\end{equation}
There is an approximate independence of kinematics on the coupling strength and mediator mass,
so a single upper limit on the \tttt cross section can be used to place constraints 
on couplings $g_{tZ'}$ and $g_{t\phi}$ as a function of masses $m_{Z'}$ and $m_{\phi}$,
respectively.
Cross sections of \tttt (normalized to SM) as a function of $g_{tZ'}$ 
and $g_{t\phi}$, for different assumptions of $m_{Z'}$ and $m_{\phi}$,
are shown in Fig.~\ref{fig:cross_section_zprimephi}. To illustrate a particular example,
the horizontal dotted line in the figures represents excluding cross sections more than
double that of the SM. These are translated into exclusions on $g_{tZ'}$ and $g_{t\phi}$
via crossing points that are projected onto the x axis.

\begin{figure}[!htbp]
    \centering
    \includegraphics[width=0.78\linewidth]{figs/ftan/plot_xsec_zprime.pdf} \\
    \includegraphics[width=0.78\linewidth]{figs/ftan/plot_xsec_phi.pdf}
    \caption{
        Cross sections of \tttt (normalized to SM) as a function of $g_{tZ'}$ (upper)
        and $g_{t\phi}$ (lower) for different assumptions of $m_{Z'}$ and $m_{\phi}$,
        respectively.
    }
    \label{fig:cross_section_zprimephi}
\end{figure}

\subsubsection{Oblique Higgs parameter}

In a universal effective field theory framework, the Higgs oblique
parameter $\hat H$, defined as the Wilson coefficient of the dimension-6
operator modifying the Higgs boson propagator, can result in deviations of the
SM \tttt cross section, as shown in Ref.~\cite{THEORY:ObliqueHiggs2019}.  These
(off-shell) deviations can be constrained to a level which is competitive with
constraints from on-shell processes.

The two main characteristic effects of this oblique parameter are
an additional term in the SM Higgs boson propagator
\begin{equation}
    P_h(p^2)\approx\frac{i}{p^2-m_h^2}-\frac{i\hat{H}}{m_h^2},
\end{equation}
and a rescaling of the fermionic higgs
couplings
\begin{equation}
    \kappa_f = 1-{\hat H}.
\end{equation}

Using the latest combined fits of ATLAS for the (on-shell) fermionic couplings,
with 80$\mathrm{fb}^{-1}$ of 13TeV data, the authors of Ref.~\cite{THEORY:ObliqueHiggs2019} find a constraint on
the oblique parameter of $\hat{H} < 0.16$ at 95\% CL.

The authors also calculate that the cross section of (off-shell) \tttt is subject to a fractional modification (with respect to the SM cross section)
at 14 TeV, given by,
\begin{equation}
    \frac{\sigma_{\hat{H}+\mathrm{SM}}}{\sigma_\mathrm{SM}} = 1 + 0.03\left(\frac{\hat{H}}{0.04}\right) + 0.15\left(\frac{\hat{H}}{0.04}\right)^2.
\end{equation}
For an oblique parameter value of 0.1, the formula predicts a doubling of the SM cross section of \tttt.

The SM model within the MadGraph~\cite{THEORY:MADGRAPH5} generator was modified to take into account the extra term in the propagator, as
well as the rescaling of the top-yukawa coupling, and the calculation is repeated
at 13\TeV. The resulting curve is shown in Fig.~\ref{fig:higgs_oblique_madgraph}.

When searching for SM \tttt and placing upper limits on the production cross
section, one can use the relative size of the upper limit with respect to
the SM prediction to exclude $\hat{H}$ values above a threshold. For example,
excluding cross sections more than double that of the SM, $\hat{H}$ values
above approximately 0.14 can be excluded. There are two important caveats
that will need to be taken into account when performing an interpretation for
$\hat{H}$. First, the kinematics of $\tttt$ will be slightly different
depending on the value of $\hat{H}$. Second, the SM process $\ttH$, which is
relevant for the $\tttt$ search, is proportional to
$y_\mathrm{t}^2=(1-\hat{H})^2$ ($\approx 0.74$ at $\hat{H}=0.14$).

\begin{figure}[!htbp]
    \centering
    \includegraphics[width=0.75\linewidth]{figs/ftan/higgs_oblique.pdf}
    \caption{
        Cross section (normalized to SM) as a function of oblique parameter $\hat{H}$.
        The green curve is a calculation from MadGraph at 13TeV, and
        the solid black curve is a cubic fit to the calculation.
    }
    \label{fig:higgs_oblique_madgraph}
\end{figure}