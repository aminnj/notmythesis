\chapter{Objects and variables}
\mytodo{make todo}

\section{Jets, MET}
\subsection{Jet identification, corrections}
\subsection{MET}
\subsection{B tagging}
% performance plot
% define njets, nb here


\section{Leptons}

\subsection{Identification}

\subsection{Isolation}
\mytodo{already reworded}

Conceptually, a lepton can be labeled isolated if the ratio of 
energy in a cone surrounding the lepton to the energy of the lepton itself
(``relative isolation'') is below a certain threshold.
However, we instead use three more sophisticated variables, improving upon this concept for robustness,
to classify leptons as isolated:
\begin{itemize}
    \item ``mini-isolation'' $\miniiso$: 
    \begin{equation}
        \miniiso = \frac{\sum_R \pt(h^\pm) - \mbox{max}(0, \sum_R \pt(h^0)+\pt(\gamma) - \rho\mathcal{A}\left(\frac{R}{0.3}\right)^2}{\pt(\ell)}.
    \end{equation}
    where $\rho$ is the event-level energy density from pileup,
    and $\sum_R\pt(h^\pm)$, $\sum_R\pt(h^0)$, $\sum_R\pt(\gamma)$ 
    are the sums of the $\pt$ of charged hadrons, neutral hadrons, and photons, respectively.
    The sum is performed within a cone of radius $R$, which depends on lepton \pt:
    \begin{equation}
        R = \frac{10}{\mbox{min}(\mbox{max}(\pt(\ell), 50), 200)}
    \end{equation}
    The effective areas $\mathcal{A}$ are constants calculated in coarse bins of
    lepton $\eta$ such that the last term of the numerator represents the
    contribution from pileup and is subtracted off.
    Requiring $\miniiso$ to be below a particular value ensures that the
    lepton is isolated.

    \item \ptratio, defined as the ratio of the lepton \pt to the
    \pt of the jet geometrically closest to, or containing, the lepton:
    \begin{equation}
        \ptratio = \frac{\pt(\ell)}{\pt(\text{jet})}
    \end{equation}
    In order to avoid an
    over-correction on prompt leptons, the application of the jet energy
    correction to mitigate is only applied on the hadronic part of the jet.
    That is, the denominator of $\ptratio$ is corrected for pileup effects
    after the lepton is subtracted out, and the lepton is subsequently added back.

    \item \ptrel:
    \begin{equation}
        \ptrel=\frac{\left|\left(\vec{p}(\text{jet})-\vec{p}(\ell)\right) 
        \times \vec{p}(\ell)\right| }{|\vec{p}(\text{jet})-\vec{p}(\ell)|}
    \end{equation}
    This variable is a measure of the relative transverse separation between the lepton 
    and matching jet. For leptons arising from the decay of B mesons, for example,
    this quantity exhibits a kinematic cutoff of a few GeV. For leptons that
    happen to overlap accidentally with jets, \ptrel compares too uncorrelated
    quantities and exhibits no kinematic cutoff, so this quantity can be large.
    This property allows us to recover leptons that would be labeled non-isolated
    by the previous two variables.
\end{itemize}

Using the above three variables, we classify a lepton
as isolated if the following boolean condition is satisfied:
\begin{equation}
  \miniiso < I_1 \wedge ( \ptratio > I_2 \vee \ptrel > I_3 )
\end{equation}
where $I_i$ threshold values depend on the lepton flavor and PU conditions
of the flavor of the lepton. These values are tabulated in \ref{tab:isoWPs}.

\begin{table}[h]
    \label{tab:isoWPs}
    \centering
    \caption{Isolation working points }
    \begin{tabular}{|l||c|c|c|c|}
        \hline
        & e/$\mu$ loose WP &  $\mu$ tight WP & e tight WP \\ \hline 
        $I_1$ & 0.4 & 0.16 (2016), 0.11 (2017/2018) & 0.12 (2016), 0.07 (2017/2018) \\
        $I_2$ & 0  & 0.76 (2016), 0.74 (2017/2018) & 0.80 (2016), 0.78 (2017/2018) \\
        $I_3$ & 0  & 7.2 (2016), 6.8 (2017/2018) & 7.2 (2016), 8.0 (2017/2018) \\ \hline
    \end{tabular}
\end{table}

\section{Trigger}
trigger plot

\section{Key variables}
mtmin njets nleps, ...
% mtmin, njets nle

\section{Event-level BDT}
% bdt for four top
describe bdt example, variables, training, result ROC

\section{Signal regions}
\mytodo{where do baseline selections go??}
