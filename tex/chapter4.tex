\chapter{Background estimation}

There are three main classes of backgrounds that survive the inclusive SUSY or
\smft analysis baseline selections: backgrounds with two or more prompt leptons
in the final state (giving a real SS pair), backgrounds with at least one nonprompt lepton,
and backgrounds that actually have an OS pair with one lepton having misreconstructed charge.

The first class of backgrounds are ``rare'' (have low cross section) and are estimated from
simulation, with appropriate correction factors and uncertainties to be discussed in the next
sections. This class can be further subdivided by the physical process:

\begin{itemize}
    \item \textbf{Diboson}: \\ \WZ, \ZZ, \Zgamma, \Wgamma, and \WpWp
    \item \textbf{Triboson}: \\ \WWW, \WWZ, \WZZ, \ZZZ, \WWgamma, and \WZgamma
    \item \textbf{Single top quark and bosons}: \\ \tgamma, \tZgamma, and \tWZ
    \item \textbf{Top quark pair and one boson}: \\ \ttW, \ttZ, \ttH, and \ttgamma
    \item \textbf{Top quark pair and two bosons}: \\ \ttWW, \ttWZ, \ttZZ, \ttWH, \ttZH, and \ttHH
    \item \textbf{Triple top quark}: \\ \ttt and \tttW
    \item \textbf{Four top quarks}: \\ \tttt
\end{itemize}

In the \smft analysis, \tttt is, of course, considered a signal instead of a background.
Contributions from the first two categories of physical processes are negligible for the
\smft analysis due to the $\Nbjets\geq 2$ requirement in the baseline selections.

The second class of background events consist of events where one of the leptons is
``nonprompt'', or more colloquially, ``fake''. That is, the fake lepton is
either a real lepton which is a decay product of a heavy flavor hadron (a
\PQb or \PQc quark), or simply a misidentified hadron. The dominant sources of fake leptons
are the large cross section processes of $\ttjets$ and $\wjets$. This class is more
tricky to only use simulation, and a data-driven method is used instead.

The last class of backgrounds consists of those with a charge-misidentified lepton, or,
more colloquially ``flips'', which essentially
converts an OS pair into a SS pair. Thus, the biggest source of this background is 
the highest cross-section OS process: Drell-Yan ($\PZ/\gamma$). Similarly to
the nonprompt background, this background uses a data-driven method instead of 
relying on completely on simulation.

\section{Rare SM}
\mytodo{todo}

\section{Nonprompt leptons}
\mytodo{todo}

\section{Charge misidentification}
\mytodo{todo}

\section{Corrections}
\mytodo{todo}

\section{Control regions}
\mytodo{todo}

\section{Systematic uncertainties}
\mytodo{todo}

